% The first argument is the \documentclass, which tells latex which template
% we're using to build this document. It's usually safe to just use "article".
\documentclass{article}

% include some packages...
\usepackage{fullpage} % change settings for a smaller margin
\usepackage{graphicx} % gives access to the \includegraphics commands
\usepackage{amsfonts}
\usepackage{float}
\usepackage{enumitem}
\usepackage{caption}
\usepackage[export]{adjustbox}
\usepackage{bookmark}
\graphicspath{{./images/}}

% tell Latex to use no paragraph indentation, but leave some space between
% paragraphs
\setlength{\parindent}{0in}
\setlength{\parskip}{0.1in}

\newcommand{\tib}[1]{\textit{\textbf{#1}}}
\newcommand{\code}[1]{\texttt{#1}}

% these commands merely set the values for the title/date/author; they don't put
% them in the document... see \maketitle below
\title{CS Department Automated Information Timeline \\ Assignment 6.3: GRASP Concepts}
\date{\today}
\author{Matthew Hays, Pawan Bhandari, Sarah Faron, Tim Klimpel}

% all document content goes between \begin{document} and \end{document}
\begin{document}

% this command actually creates the title/date/author in the document
\maketitle
\newpage
\tableofcontents
\listoffigures
\newpage
\section{Introduction}
\subsection{Purpose}
The purpose of this assignment is to work as a team and collaboratively to identify General Responsibility Assignment Software Patterns (GRASP) in the design of the CS Department Automated Information Timeline project. The team met multiple times over the course of a few days to work together and identify the GRASP concepts. Section 2 of this document consists of the identified GRASP concepts and section 3 contains the annotated sequence diagrams.

\subsection{GRASP concepts}
\subsection{Annotated Sequence diagrams}
\end{document}