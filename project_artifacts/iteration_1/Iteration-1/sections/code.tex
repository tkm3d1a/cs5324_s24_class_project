For the initial iteration submission, we have provided an implemented domain using the
Spring JPA framework. The following items were considered in this iteration. The code base will continue to evolve over subsequent iterations.
\begin{itemize}
    \item Version allows for optimistic locking during database transactional contexts to ensure that a stale persistence does not occur.

    \item Created at and updated at values use Spring auditing to add the appropriate system time to our entities upon submission.

    \item Enumerations were provided as separate classes to promote reusability and extensibility within the code base.

    \item Care was given in assigning relationship owning entities to allow for succinct operations in persistence and modification.

    \item At this time, no cascading operation have been determined. This is to be provided later as realization of these operations will become more available while development continues.

    \item Special consideration was given to the Notification entity to ensure that it may have one and only one relationship with the following differing entity types: Post, Event, Media.

    \item Custom user implementation of UserDetails and UserDetailsService (from Spring security) was used to create the base user class. This was chosen since it will provide roles tied to logged in users and can be fetched to determine which options are available to a logged in user based on the granted authorities for that role.
\end{itemize}