% The first argument is the \documentclass, which tells latex which template
% we're using to build this document. It's usually safe to just use "article".
\documentclass{article}

% include some packages...
\usepackage{fullpage} % change settings for a smaller margin
\usepackage{graphicx} % gives access to the \includegraphics commands
\usepackage{amsfonts}
\usepackage{float}
\usepackage{enumitem}
\graphicspath{{./images/}}

% tell Latex to use no paragraph indentation, but leave some space between
% paragraphs 
\setlength{\parindent}{0in}
\setlength{\parskip}{0.1in}

\newcommand{\tib}[1]{\textit{\textbf{#1}}}
\newcommand{\code}[1]{\texttt{#1}}

% these commands merely set the values for the title/date/author; they don't put
% them in the document... see \maketitle below
\title{CS Department Automated Information Timeline. \\ Assignment 3.1: Domain Model}
\date{\today}
\author{Matthew Hays, Pawan Bhandari, Sarah Faron, Tim Klimpel}

% all document content goes between \begin{document} and \end{document}
\begin{document}

% this command actually creates the title/date/author in the document
\maketitle
\newpage
\tableofcontents
\listoffigures
\newpage

\section{Identifying Candidate Concepts}
According to C. Larman, The three strategies to identify candidate concepts are as follows:
\begin{enumerate}
    \item \textbf{Reuse or modify existing models}
    \item \textbf{Use a category list}
    \item \textbf{Identify noun phrases}
\end{enumerate}
\section{Concepts, Attributes, and Associations}
\section{Domain Model}
\subsection{Domain Model (Draft)}
\subsection{Discussion of domain model draft}
\subsection{Domain Model (Final)}
\end{document}